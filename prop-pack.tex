\documentclass[12pt]{article}

\usepackage[scaled=0.9]{beramono}
\usepackage{xcolor}

\usepackage[T1]{fontenc}
\usepackage{mathpazo}
\linespread{1.05}
\usepackage[T1,small,euler-digits]{eulervm}

\usepackage[utf8x]{inputenc}
 \usepackage{hyperref}
\hypersetup{linktocpage}
\hypersetup{
    colorlinks,
    citecolor=amethyst,
    filecolor=black,
    linkcolor=blue(munsell),
    urlcolor=blue(munsell)
}
\usepackage[T1]{fontenc}
\usepackage[]{amsmath,amssymb}

\definecolor{amethyst}{rgb}{0.6, 0.4, 0.8}
\definecolor{antiquefuchsia}{rgb}{0.57, 0.36, 0.51}
\definecolor{ballblue}{rgb}{0.13, 0.67, 0.8}
\definecolor{blue(munsell)}{rgb}{0.0, 0.5, 0.69}
\definecolor{blue(pigment)}{rgb}{0.2, 0.2, 0.6}
\definecolor{brightmaroon}{rgb}{0.76, 0.13, 0.28}
\definecolor{bondiblue}{rgb}{0.0, 0.58, 0.71}
\definecolor{cadet}{rgb}{0.33, 0.41, 0.47}
\definecolor{ashgrey}{rgb}{0.7, 0.75, 0.71}
\definecolor{amber}{rgb}{1.0, 0.75, 0.0}
\definecolor{aliceblue}{rgb}{0.94, 0.97, 1.0}
\usepackage{listings,chngcntr}

\lstset{
    aboveskip=10pt
  , backgroundcolor=\color{aliceblue}
  , basicstyle=\ttfamily
  , belowskip=10pt
  , frame=trbl
  , keywords=[2]{axiom, conjecture,inference, negated_conjecture, plain}
  , keywords=[3]{fof, cnf}
  , keywords=[4]{canonicalize, resolve, conjunct,
    strip, simplify, negate, clausify}
  , keywordstyle=[2]\color{cadet}%\underbar,%\bfseries
  , keywordstyle=[3]\color{bondiblue}\bfseries
  , keywordstyle=[4]\bfseries
  , showstringspaces=false
}

\usepackage[]{graphicx}

\usepackage{array}
\usepackage[margin=1in]{geometry}

\usepackage{url}
\usepackage{float}
\usepackage{syntax}
% \setlength{\grammarparsep}{20pt plus 1pt minus 1pt} % increase separation between rules
\setlength{\grammarindent}{4em} % increase separation between LHS/RHS

\setlength\parindent{0pt} % Removes all indentation from paragraphs

\newcommand{\textdf}[1]{\textbf{\textsf{#1}}\index{#1}}
\newcommand{\tptp}[0]{\texorpdfstring{\texttt{TPTP}}\ \ }
\newcommand{\miss}[0]{{\colorbox{amber}{\textdf{?}}}}
\newcommand{\ttext}[1]{\texttt{\color{black} #1}}
\newcommand{\shell}[1]{\texttt{\color{black} #1}}

\newcommand{\problemtptp}[3][c]{\lstinputlisting[caption=#2, firstline=#3, label=custom#1]{#2}}
\newcommand{\solutiontstp}[2][c]{\lstinputlisting[caption=#2, firstline=5, label=custom#1]{#2}}

\setlength{\topmargin}{-.35 in}
\setlength{\oddsidemargin}{0in}
\setlength{\evensidemargin}{0in}
\setlength{\textheight}{9.in}
\setlength{\textwidth}{6.7in}

\begin{document}
\counterwithin{lstlisting}{section}

\hrulefill\ \ Version: \today\\

\vspace{3mm}
\begin{center}
{\Large Prop-Pack Repository\footnote{
\href{https://github.com/jonaprieto/prop-pack}{{\color{blue(munsell)}
\texttt{https://github.com/jonaprieto/prop-pack}}}}
}\\
\textbf{Jonathan Prieto-Cubides}\\ %You should put your name here
\href{mailto:jprieto9@eafit.edu.co}{{\color{blue(munsell)}\texttt{jprieto9@eafit.edu.co}}}
\end{center}
\vspace{0.1 cm}

\tableofcontents

\section{\tptp Format}
The \tptp format is a precise and accurate language to describe without ambiguities problems written in a logic. It attempts to be a standard for all Automatic Theorems Provers (ATP). The syntax of the TPTP format is based on the Prolog language and endeavors to be processed with a similar parser for such a language\footnote{\tptp v6.4.0~\url{http://www.cs.miami.edu/~tptp/TPTP/SyntaxBNF.html}}.\par

The following is an example taken from the \tptp
Problems repository\footnote{\url{http://www.cs.miami.edu/~tptp/cgi-bin/SeeTPTP?Category=Problems}}.

\begin{lstlisting}
fof(a1, axiom, a).
cnf(an_2, axiom, (is_a_theorem(or(not(or(X,Y)) , or(Y,X))) )).
\end{lstlisting}

We present a subset of \tptp that use for first-order logic the keyword \ttext{fof} and the keyword \ttext{cnf} to use the clause normal form. Their grammars are as follows.

\label{syntax:fof}
\begin{grammar}
<cnf_annotated> ::= \textbf{cnf}(<name>,<formula_role>,<cnf_formula>,<annotations>).

<fof_annotated> ::= \textbf{fof}(<name>,<formula_role>,<fof_formula>,<annotations>).
\end{grammar}

The last parameter \ttext{annotations} is optional and the others parameters like
\ttext{name}, \ttext{formula_role}, \ttext{fof_formula}, \ttext{cnf_formula} have
the following definition.

\begin{grammar}
<name> ::= <atomic_word> | <integer>

<formula_role> ::= {\bf axiom} | hypothesis | definition | assumption
\alt lemma | theorem | corollary | {\bf conjecture}
\alt {\bf negated_conjecture} | {\bf plain }| type | unknown
\end{grammar}
% For the clausal normal form, we have the following excerpt.

\begin{figure}[!ht]
\centering
\label{syntax:cnfformula}
\begin{grammar}
<cnf_formula> ::= (<disjunction>) | <disjunction>

<disjunction> ::= <literal> | <disjunction> <vline> <literal>

<literal> ::= <atomic_formula> | ~ <atomic_formula> | <fol_infix_unary>
\end{grammar}
\caption{Grammar for CNF formulas}
\end{figure}

Some of the ATP systems can process problems formatted
in TPTP syntax for their input and output. In addition, there is an extension
version of TPTP to present solutions\footnote{\url{http://www.cs.miami.edu/~tptp/TSTP/}},
the TSTP syntax. Also, the TSTP (Thousands of Solutions from Theorem Provers) is
for a library of solutions to test problems for automated theorem proving (ATP)
systems. It has solutions to problems from the TPTP Problem Library.\par

\section{TPTP Problems}

\subsection{Basic}
\problemtptp[basic-1.tptp]{problems/basic/basic-1.tptp}{2}
\problemtptp[basic-2.tptp]{problems/basic/basic-2.tptp}{2}
\problemtptp[basic-3.tptp]{problems/basic/basic-3.tptp}{2}
\problemtptp[basic-4.tptp]{problems/basic/basic-4.tptp}{2}

\subsection{Conjunction}
\problemtptp[conj-1.tptp]{problems/conjunction/conj-1.tptp}{2}
\problemtptp[conj-2.tptp]{problems/conjunction/conj-2.tptp}{2}
\problemtptp[conj-3.tptp]{problems/conjunction/conj-3.tptp}{2}
\problemtptp[conj-4.tptp]{problems/conjunction/conj-4.tptp}{2}
\problemtptp[conj-5.tptp]{problems/conjunction/conj-5.tptp}{2}
\problemtptp[conj-6.tptp]{problems/conjunction/conj-6.tptp}{2}

\subsection{Implication}
\problemtptp[impl-1.tptp]{problems/implication/impl-1.tptp}{2}
\problemtptp[impl-2.tptp]{problems/implication/impl-2.tptp}{2}
\problemtptp[impl-3.tptp]{problems/implication/impl-3.tptp}{2}
\problemtptp[impl-4.tptp]{problems/implication/impl-4.tptp}{2}
\problemtptp[impl-5.tptp]{problems/implication/impl-5.tptp}{2}
\problemtptp[impl-6.tptp]{problems/implication/impl-6.tptp}{2}
\problemtptp[impl-7.tptp]{problems/implication/impl-7.tptp}{2}
\problemtptp[impl-8.tptp]{problems/implication/impl-8.tptp}{2}
\problemtptp[impl-9.tptp]{problems/implication/impl-9.tptp}{2}
\problemtptp[impl-10.tptp]{problems/implication/impl-10.tptp}{2}
\problemtptp[impl-11.tptp]{problems/implication/impl-11.tptp}{2}
\problemtptp[impl-12.tptp]{problems/implication/impl-12.tptp}{2}
\problemtptp[impl-13.tptp]{problems/implication/impl-13.tptp}{2}
\problemtptp[impl-14.tptp]{problems/implication/impl-14.tptp}{2}
\problemtptp[impl-15.tptp]{problems/implication/impl-15.tptp}{2}
\problemtptp[impl-16.tptp]{problems/implication/impl-16.tptp}{2}
\problemtptp[impl-17.tptp]{problems/implication/impl-17.tptp}{2}
\problemtptp[impl-18.tptp]{problems/implication/impl-18.tptp}{2}

\subsection{Disjunction}
%
\problemtptp[disj-1.tptp]{problems/disjunction/disj-1.tptp}{2}
\problemtptp[disj-2.tptp]{problems/disjunction/disj-2.tptp}{2}
\problemtptp[disj-3.tptp]{problems/disjunction/disj-3.tptp}{2}
\problemtptp[disj-4.tptp]{problems/disjunction/disj-4.tptp}{2}
\problemtptp[disj-5.tptp]{problems/disjunction/disj-5.tptp}{2}
\problemtptp[disj-6.tptp]{problems/disjunction/disj-6.tptp}{2}
\problemtptp[disj-7.tptp]{problems/disjunction/disj-7.tptp}{2}
\problemtptp[disj-8.tptp]{problems/disjunction/disj-8.tptp}{2}
\problemtptp[disj-9.tptp]{problems/disjunction/disj-9.tptp}{2}
\problemtptp[disj-10.tptp]{problems/disjunction/disj-10.tptp}{2}
\problemtptp[disj-11.tptp]{problems/disjunction/disj-11.tptp}{2}
\problemtptp[disj-12.tptp]{problems/disjunction/disj-12.tptp}{2}
\problemtptp[disj-13.tptp]{problems/disjunction/disj-13.tptp}{2}
\problemtptp[disj-14.tptp]{problems/disjunction/disj-14.tptp}{2}
\problemtptp[disj-15.tptp]{problems/disjunction/disj-15.tptp}{2}
%
\subsection{Biimplication}

\problemtptp[bicond-1.tptp]{problems/biconditional/bicond-1.tptp}{2}
\problemtptp[bicond-2.tptp]{problems/biconditional/bicond-2.tptp}{2}
\problemtptp[bicond-3.tptp]{problems/biconditional/bicond-3.tptp}{2}
\problemtptp[bicond-4.tptp]{problems/biconditional/bicond-4.tptp}{2}
\problemtptp[bicond-5.tptp]{problems/biconditional/bicond-5.tptp}{2}
\problemtptp[bicond-6.tptp]{problems/biconditional/bicond-6.tptp}{2}
\problemtptp[bicond-7.tptp]{problems/biconditional/bicond-7.tptp}{2}
\problemtptp[bicond-8.tptp]{problems/biconditional/bicond-8.tptp}{2}
\problemtptp[bicond-9.tptp]{problems/biconditional/bicond-9.tptp}{2}
\problemtptp[bicond-10.tptp]{problems/biconditional/bicond-10.tptp}{2}
\problemtptp[bicond-11.tptp]{problems/biconditional/bicond-11.tptp}{2}
\problemtptp[bicond-12.tptp]{problems/biconditional/bicond-12.tptp}{2}
\problemtptp[bicond-13.tptp]{problems/biconditional/bicond-13.tptp}{2}


\subsection{Negation}

\problemtptp[neg-1.tptp]{problems/negation/neg-1.tptp}{2}
\problemtptp[neg-2.tptp]{problems/negation/neg-2.tptp}{2}
\problemtptp[neg-3.tptp]{problems/negation/neg-3.tptp}{2}
\problemtptp[neg-4.tptp]{problems/negation/neg-4.tptp}{2}
\problemtptp[neg-5.tptp]{problems/negation/neg-5.tptp}{2}
\problemtptp[neg-6.tptp]{problems/negation/neg-6.tptp}{2}
\problemtptp[neg-7.tptp]{problems/negation/neg-7.tptp}{2}
\problemtptp[neg-8.tptp]{problems/negation/neg-8.tptp}{2}
\problemtptp[neg-9.tptp]{problems/negation/neg-9.tptp}{2}
\problemtptp[neg-10.tptp]{problems/negation/neg-10.tptp}{2}
\problemtptp[neg-11.tptp]{problems/negation/neg-11.tptp}{2}
\problemtptp[neg-12.tptp]{problems/negation/neg-12.tptp}{2}
\problemtptp[neg-13.tptp]{problems/negation/neg-13.tptp}{2}
\problemtptp[neg-14.tptp]{problems/negation/neg-14.tptp}{2}
\problemtptp[neg-15.tptp]{problems/negation/neg-15.tptp}{2}
\problemtptp[neg-16.tptp]{problems/negation/neg-16.tptp}{2}
\problemtptp[neg-17.tptp]{problems/negation/neg-17.tptp}{2}
\problemtptp[neg-18.tptp]{problems/negation/neg-18.tptp}{2}
\problemtptp[neg-19.tptp]{problems/negation/neg-19.tptp}{2}
\problemtptp[neg-20.tptp]{problems/negation/neg-20.tptp}{2}
\problemtptp[neg-21.tptp]{problems/negation/neg-21.tptp}{2}
\problemtptp[neg-22.tptp]{problems/negation/neg-22.tptp}{2}
\problemtptp[neg-23.tptp]{problems/negation/neg-23.tptp}{2}
\problemtptp[neg-24.tptp]{problems/negation/neg-24.tptp}{2}
\problemtptp[neg-25.tptp]{problems/negation/neg-25.tptp}{2}
\problemtptp[neg-26.tptp]{problems/negation/neg-26.tptp}{2}
\problemtptp[neg-27.tptp]{problems/negation/neg-27.tptp}{2}
\problemtptp[neg-28.tptp]{problems/negation/neg-28.tptp}{2}
\problemtptp[neg-29.tptp]{problems/negation/neg-29.tptp}{2}
\problemtptp[neg-30.tptp]{problems/negation/neg-30.tptp}{2}
\problemtptp[neg-31.tptp]{problems/negation/neg-31.tptp}{2}
\problemtptp[neg-32.tptp]{problems/negation/neg-32.tptp}{2}

\section{References}

\begin{thebibliography}{9}
\bibitem{alastair2017}
Car Alastair (2017).  \textit{The Natural Deduction Pack}. Revisited edition, Feb. 2017.
\end{thebibliography}
\end{document}
